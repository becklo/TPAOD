\documentclass[a4paper, 10pt, french]{article}
% Préambule; packages qui peuvent être utiles
   \RequirePackage[T1]{fontenc}        % Ce package pourrit les pdf...
   \RequirePackage{babel,indentfirst}  % Pour les césures correctes,
                                       % et pour indenter au début de chaque paragraphe
   \RequirePackage[utf8]{inputenc}   % Pour pouvoir utiliser directement les accents
                                     % et autres caractères français
   % \RequirePackage{lmodern,tgpagella} % Police de caractères
   \textwidth 17cm \textheight 25cm \oddsidemargin -0.24cm % Définition taille de la page
   \evensidemargin -1.24cm \topskip 0cm \headheight -1.5cm % Définition des marges
   \RequirePackage{latexsym}                  % Symboles
   \RequirePackage{amsmath}                   % Symboles mathématiques
   \RequirePackage{tikz}   % Pour faire des schémas
   \RequirePackage{graphicx} % Pour inclure des images
   \RequirePackage{listings} % pour mettre des listings
% Fin Préambule; package qui peuvent être utiles

\title{Rapport de TP 4MMAOD : Génération d'ABR optimal}
\author{
BECK Loula (groupe 1) 
\\ RÜHL Quentin (groupe 1) 
}

\begin{document}

\maketitle

%%%%%%%%%%%%%%%%%%%%%%%%%%%%%%%%%%%%%%%%%%%%%%
\paragraph{\em Préambule}
{\em \begin{itemize} 
   \item Compléter ce patron de rapport en supprimant toutes les phrases en italique\,: elles ne doivent pas apparaître dans le rapport pdf.
   \item Il sera attribué {\bf 1 point} pour la qualité globale du rapport\,: présentation, concision et clarté de l'argumentation.
\end{itemize}
}

%%%%%%%%%%%%%%%%%%%%%%%%%%%%%%%%%%%%%%%%%%%%%%
\section{Principe de notre  programme (1 point)}
 Tout d'abord, il s'agira pour notre programme de lire les benchmark différents. Pour ce faire, nous avons implémenté le fichier {fileReader.c}.
 Dans ce fichier, on ouvrira le fichier demandé puis grâce à la fonction C fscanf() on lira les entiers présents. On les placera 
 dans un tableau bidimensionnel avec le nombre d'occurence qui leur correspond. Toutes les valeurs d'entier sont initialisées à -1 dans le tableau, les occurences, à 0.
 On créée ensuite une structure COUPLE qui se compose la valeur d'un élément {\em el} et de sa probabilité {\em prob}.
 On crée un tableau de COUPLE trié en fonction de la valeur des éléments. Ce tableau pourra être utilisé pour la suite pour la construction de l'arbre.

%%%%%%%%%%%%%%%%%%%%%%%%%%%%%%%%%%%%%%%%%%%%%%
\section{Analyse du coût théorique (2 points)}
{\em Donner ici l'analyse du coût théorique de votre programme en fonction du nombre $n$ d'éléments dans le dictionnaire.
 Pour chaque coût, donner la formule qui le caractérise (en justifiant brièvement pourquoi cette formule correspond à votre programme), 
 puis l'ordre du coût en fonction de $n$ en notation $\Theta$ de préférence, sinon $O$.}

  \subsection{Nombre  d'opérations en pire cas\,: }
    \paragraph{Justification\,: }
    {\em La justification peut être par exemple de la forme: \\ 
       "Le programme itératif contient les boucles $k_1=...$, $k_2= ...$ etc correspondant à la somme 
      $$T(n_1, n_2, c_1, c_2) = \sum_{k_1=...}^{...} ... \sum ... + \sum_{i=...}^{...} ...$$ 
      somme que nous avons calculée (ou majorée) par la technique de  ... " \\
      ou  encore\,:  \\
      "les appels récursifs du programme permettent de modéliser son coût par le système d'équations aux récurrences 
      $$T(k_1, k_2) = ...  \mbox{~avec~les~conditions~initiales~....~} $$
      Le coût indiqué est obtenu en résolvant ce système par la méthode de  .... "
    } 
  \subsection{Place mémoire requise\,: }
    \paragraph{Justification\,: }

  \subsection{Nombre de défauts de cache sur le modèle CO\,: }
    \paragraph{Justification\,: }


%%%%%%%%%%%%%%%%%%%%%%%%%%%%%%%%%%%%%%%%%%%%%%
\section{Compte rendu d'expérimentation (2 points)}
  \subsection{Conditions expérimentaless}
    \subsubsection{Description synthétique de la machine\,:} 
      Les caractéristiques de la machine utilisée pour les tests sont :
      \begin{itemize}
       \item Processeur Intel Core i3
       \item Fréquence 2GHz
       \item Mémoire 4Gb
       \item OS Ubuntu 16.04 LTS
      \end{itemize}
      Le seul processus user qui tourne en même temps sur la machine est le bash lors de la réalisation des tests.

    \subsubsection{Méthode utilisée pour les mesures de temps\,: } 
      Pour mesurer le temps, nous avons choisi d'utiliser la fonction clock() de la bibliothèque {\em time.h}. Avant l'appel à la fonction
      permettant la construction de l'arbre, on relève une première fois le temps processeur tStart.
      On relève ensuite le temps juste après son execution tEnd. 
      Le temps total d'execution est donc la différence de ces deux temps divisé par le nombre de tour d'horloge du processeur par sec CLOCK\_PER\_SEC.
      Les 5 mesures sont éxécutées de manières consécutives pour chaque benchmark. 

  \subsection{Mesures expérimentales}
    Les mesures efféctuées nous permettent de construire le tableau suivant :
    \begin{figure}[h]
      \begin{center}
        \begin{tabular}{|l||r||r|r|r||}
          \hline
          \hline
            & coût         & temps     & temps   & temps \\
            & du patch     & min       & max     & moyen \\
          \hline
          \hline
            benchmark1 &      &     &     &     \\
          \hline
            benchmark2 &      &     &     &     \\
          \hline
            benchmark3 &      &     &     &     \\
          \hline
            benchmark4 &      &     &     &     \\
          \hline
            benchmark5 &      &     &     &     \\
          \hline
            benchmark6 &      &     &     &     \\
          \hline
          \hline
        \end{tabular}
        \caption{Mesures des temps minimum, maximum et moyen de 5 exécutions pour les 6 benchmarks.}
        \label{table-temps}
      \end{center}
    \end{figure}

\subsection{Analyse des résultats expérimentaux}
{\em Donner  une réponse justifiée  à la question\,: 
              les  temps mesurés correspondent ils  à votre analyse théorique (nombre d’opérations et défauts de cache) ?
}


\end{document}
%% Fin mise au format

